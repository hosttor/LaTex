% this is a comment sign

\documentclass{class} %Use to Specify Project at the begining of document

Docunemt Class Options
[10pt/11pt/12pt] %Font Size
[letterpaper/a4paper] %Paper Size
[twocolumn] %Use two columns
[twoside] %Set margins for two-sided
[landscape] %Landscape orientation. Must use dvips

Packages: Use before \begin{document}. Usage: \usepackage{package}

fullpage %Use 1 inch margins.
anysize %Set margins: \marginsize{l}{r }{t}{b}.
multicol %Use n columns: \begin{multicols}{n}.
latexsym %Use L A TEX symbol font.
graphicx %Show image: \includegraphics[width=x ]{file}.
url %Insert URL: \url{http://. . . }.

Meta Data
\author{text} %Author of document.
\title{text} %Title of document.
\date{text}
\maketitle

Miscellaneous
\pagestyle{empty}

\begin{document} %Use to start contents
\end{document} %Use to end the document

Document structure
\part{title}
\subsubsection{title}
\chapter{title}
\paragraph{title}
\section{title}
\subparagraph{title}
\subsection{title} %Section commands can be followed with an *, like
\section*{title}, %to supress heading numbers.
\setcounter{secnumdepth}{x} %supresses heading numbers of depth > x, where chapter has depth 0.

Text environments
\begin{comment} %Comment block (not printed).
\begin{quote} %Indented quotation block.
\begin{quotation} %Like quote with indented paragraphs.
\begin{verse} %Quotation block for verse.

Lists
\begin{enumerate} %Numbered list
\begin{itemize} %Bulleted list
\begin{description} %Description list
\item text %Add an item
\item[x ] text %Use x instead of normal bullet or number. Required for descriptions

References
\label{marker} %Set a marker for cross-reference, often of the form \label{sec:item}.
\ref{marker } %Give section/body number of marker.
\pageref{marker } %Give page number of marker.
\footnote{text} %Print footnote at bottom of page.

Floating bodies
\begin{table}[place] %Add numbered table.
\begin{figure}[place] %Add numbered figure.
\begin{equation}[place] %Add numbered equation.
\caption{text} %Caption for the body. The place is a list valid placements for the body. t=top,h=here, b=bottom, p=separate page, !=place even if ugly. Captions and label markers should be within the environment.

Text properties
Font face
%Command %Declaration %Effect
\textrm{text} {\rmfamily text} %Roman family
\textsf{text} {\sffamily text} %Sans serif family
\texttt{text} {\ttfamily text} %Typewriter family
\textmd{text} {\mdseries text} %Medium series
\textbf{text} {\bfseries text} %Bold series
\textup{text} {\upshape text} %Upright shape
\textit{text} {\itshape text} %Italic shape
\textsl{text} {\slshape text} %Slanted shape
\textsc{text} {\scshape text} %Small Caps shape
\emph{text} {\em text} %Emphasized
\textnormal{text}{\normalfont text} %Document font
\underline{text} %Underline
The command (tttt) %form handles spacing better than the
declaration (tttt) %form.

Font size
\tiny %tiny
\scriptsize %scriptsize
\footnotesize %footnotesize
\small %small
\normalsize %normalsize
\large %large
\Large %large
\LARGE %extra lare
\huge %huge
\Huge %extra huge

Verbatim Text
\begin{verbatim} %Verbatim environment
\begin{verbatim*} %Spaces are shown as
\verb!text! %Text between the delimitin

Miscellaneous
\linespread{x} %changes the line spacing by the multiplier x

Text-mode symbols
List:& % \& \$ \% ... | # \_ \^{} \~{} ˆ  ̃ \ldots \textbar \# • \ § \textbullet \tex

Line and page breaks
\\ %Begin new line without new paragraph.
\\* %Prohibit pagebreak after linebreak.
\kill %Don’t print current line.
\pagebreak %Start new page.
\noindent %Do not indent current line.

Miscellaneous
\today
$\sim$ %Prints ∼ instead of \~{}, which makes  ̃
\@. %Indicate that the . ends a sentence when following an uppercase letter
\hspace{l} %Horizontal space of length l (Ex: l = 20pt).
\vspace{l} %Vertical space of length l.
\rule{w}{h} %Line of width w and height h.

Tabular environments
tabbing environment
\= %Set tab stop.
\> %Go to tab stop.
Tab stops can be set on “invisible” lines with \kill at the end
of the line. Normally \\ is used to separate lines.

tabular environment
\begin{array}[pos]{cols}
\begin{tabular}[pos]{cols}
\begin{tabular*}{width}[pos]{cols}
tabular column specification
l %Left-justified column.
c %Centered column.
r %Right-justified column.
p{width} %Same as \parbox[t]{width}.
@{decl} %Insert decl instead of inter-column space.
| %Inserts a vertical line between columns

tabular elements
\hline %Horizontal line between rows.
\cline{x-y} %Horizontal line across columns x through y.
\multicolumn{n}{cols}{text} %A cell that spans n columns, with cols column specification.

Math mode
%To use math mode, surround text with $ or \begin{equation}.
x %Subscript
^{x} %Superscript x
\frac{x}{y} %x divided by y
\sqrt[n]{x} %square root of x by n
_{x}
\sum_{k=1}^n
\prod_{k=1}^n

Math-mode symbols
≤ \leq
× \times
◦ ^{\circ}
∞ \infty
⊃ \supset
⊂ \subset
∪ \cup
a  ̇ \dot a
α \alpha
\epsilon
θ \theta
λ \lambda
π \pi
υ \upsilon
ω \omega
Λ \Lambda
Υ \Upsilon
≥
÷
◦
¬
∀
∃
∩
a
ˆ
β
ζ
ι
μ
ρ
φ
Γ
Ξ
Φ
\geq
\div
\circ
\neg
\forall
\exists
\cap
\hat a
\beta
\zeta
\iota
\mu
\rho
\phi
\Gamma
\Xi
\Phi
= \neq
± \pm
\prime
∧ \wedge
∈ \in
∈
/ \notin
| \mid
a
 ̄ \bar a
γ \gamma
η \eta
κ \kappa
ν \nu
σ \sigma
χ \chi
∆ \Delta
Π \Pi
Ψ \Psi
≈
·
···
∨
→
⇒
⇔
a
 ̃
δ
ε
θ
ξ
τ
ψ
Θ
Σ
Ω
\approx
\cdot
\cdots
\vee
\rightarrow
\Rightarrow
\Leftrightarrow
\tilde a
\delta
\varepsilon
\vartheta
\xi
\tau
\psi
\Theta
\Sigma
\Omega

Citation types
tabbing environment Full author list and year. (Watson and Crick 1953)
\citeA{key} %Full author list. (Watson and Crick)
\citeN{key} %Full author list and year. Watson and Crick(1953)
\shortcite{key} %Abbreviated author list and year. ?
\shortciteA{key} %Abbreviated author list. ?
\shortciteN{key} %Abbreviated author list and year. ?
\citeyear{key} %Cite year only. (1953)
All the above have an NP variant without parentheses; Ex.
\citeNP.

BibTEX entry types
@article %Journal or magazine article.
@book %Book with publisher.
@booklet %Book without publisher.
@conference %Article in conference proceedings.
@inbook %A part of a book and/or range of pages.
@incollection %A part of book with its own title.
@misc %If nothing else fits.
@phdthesis %PhD. thesis.
@proceedings %Proceedings of a conference.
@techreport %Tech report, usually numbered in series.
@unpublished %Unpublished.

BibTEX fields
address %Address of publisher. Not necessary for major publishers.
author %Names of authors, of format ....
booktitle %Title of book when part of it is cited.
chapter %Chapter or section number.
edition %Edition of a book.
editor %Names of editors.
institution %Sponsoring institution of tech. report.
journal %Journal name.
key %Used for cross ref. when no author.
month %Month published. Use 3-letter abbreviation.
note %Any additional information.
number %Number of journal or magazine.
organization %Organization that sponsors a conference.
pages %Page range (2,6,9--12).
publisher %Publisher’s name.
school %Name of school (for thesis).
series %Name of series of books.
title %Title of work.
type %Type of tech. report, ex. “Research Note”.
volume %Volume of a journal or book.
year %Year of publication.
Not all fields need to be filled. See example below.


